\documentclass[
	12pt,				% tamanho da fonte		
	oneside,
	a4paper,			% tamanho do papel.
	chapter=TITLE,
	%sumario=tradicional,
	english,			% idioma adicional
	brazil,				% idioma principal do documento
	]{abntex2}

% ---
% PACOTES
% ---
\usepackage{lmodern}			% Usa a fonte Latin Modern
\usepackage[T1]{fontenc}		% Selecao de codigos de fonte.
\usepackage[utf8]{inputenc}		% Codificacao do documento (conversão automática dos acentos)
\usepackage{indentfirst}		% Indenta o primeiro parágrafo de cada seção.
\usepackage{color}				% Controle das cores
\usepackage{graphicx}			% Inclusão de gráficos
\graphicspath{{figuras/},{figuras/Intro/},{figuras/CAP2/},{figuras/CAP3/},{figuras/CAP4/},{figuras/CAP5/}}
\usepackage{microtype} 			% para melhorias de justificação
\usepackage{booktabs}
\usepackage{float} 				% Set posição da figura
\usepackage[bottom]{footmisc} 
\usepackage{subfig} 			% Inserir subfiguras
\usepackage[table,xcdraw]{xcolor} 	% Cor de preenchimento das tabelas
\usepackage{multirow} 			%mesclar cel em tabelas
\usepackage{verbatim}			%Inserir codigos fontes e comentários em massa
\usepackage[brazilian,hyperpageref]{backref}
\usepackage[alf]{lib/abntex2cite}
\usepackage{lipsum}				% para geração de dummy text
\usepackage{amsmath}
\usepackage[bottom]{footmisc}
\usepackage{footnote}
%\usepackage{fnpos}
%\usepackage{ftnxtra}
\usepackage{listings} 			%Inserir códigos fontes
\usepackage{rotating} 			%Rotação de páginas
\usepackage{placeins}			% Forçar o posicionamento da figura
\usepackage{siunitx} 
\usepackage{listings}
\usepackage[top=3cm, bottom=2cm, left=3cm, right=2cm]{geometry} % Margens


% ---
% Informações de dados para CAPA e FOLHA DE ROSTO
% ---

\titulo{{\normalfont \textbf{PROPOSTA DE UMA FERRAMENTA DE BAIXO CUSTO PARA O MONITORAMENTO DE ENERGIA RESIDENCIAL}}}
\autor{JONATHAN DA SILVA BRAGA}
\local{NATAL -- RN}
\data{JUNHO DE 2018}
\instituicao{
  UNIVERSIDADE FEDERAL DO RIO GRANDE DO NORTE
  \par
  CENTRO DE TECNOLOGIA
  \par
  GRADUAÇÃO EM ENGENHARIA MECATRÔNICA

}
\tipotrabalho{Relatório técnico}

\preambulo{Trabalho de Conclusão de Curso de Engenharia Mecatrônica da Universidade Federal do Rio Grande do Norte, apresentado como requisito parcial para a obtenção do grau de Bacharel em Engenharia Mecatrônica
\newline 
\newline 
Orientador: Carlos Manuel Dias Viegas}

% Configurações de aparência do PDF final
% alterando o aspecto da cor azul
\definecolor{blue}{RGB}{41,5,195}
% informações do PDF
\makeatletter
\hypersetup{
     	%pagebackref=true,
		pdftitle={\@title}, 
		pdfauthor={\@author},
    	pdfsubject={\imprimirpreambulo},
	    pdfcreator={LaTeX with abnTeX2},
		pdfkeywords={abnt}{latex}{abntex}{abntex2}{relatório técnico}, 
		colorlinks=true,       		% false: boxed links; true: colored links
    	linkcolor=black,          	% color of internal links
    	citecolor=black,        		% color of links to bibliography
    	filecolor=magenta,      		% color of file links
		urlcolor=blue,
		bookmarksdepth=4
}
\makeatother
% Espaçamentos entre linhas e parágrafos 
\setlength{\parindent}{1.25cm} % Tamanho do parágrafo
\setlength{\parskip}{0.2cm}	% Controle do espaçamento entre um parágrafo e outro

%\onelineskip % Controle do espaçamento entre um parágrafo e outro

\makeindex % compila o indice

% Início do documento
% ----
\begin{document}

% Seleciona o idioma do documento (conforme pacotes do babel)
%\selectlanguage{english}
\selectlanguage{brazil}

% Retira espaço extra obsoleto entre as frases.
\frenchspacing 

% ----------------------------------------------------------
% ELEMENTOS PRÉ-TEXTUAIS
% ----------------------------------------------------------
\pretextual

% Capa
%\imprimircapa
\begin{titlingpage}
	\begin{center}
		UNIVERSIDADE FEDERAL DO RIO GRANDE DO NORTE \\
		CENTRO DE TECNOLOGIA \\
		GRADUAÇÃO EM ENGENHARIA MECATRÔNICA

		\vspace*{1cm}

		JONATHAN DA SILVA BRAGA

		\vspace*{\fill}
		\textbf{PROPOSTA DE UMA FERRAMENTA DE BAIXO CUSTO PARA O MONITORAMENTO DE ENERGIA RESIDENCIAL}
		\vspace*{\fill}

		NATAL -- RN\\
		JUNHO DE 2018
		\vspace*{0.5cm}

	\end{center}
\end{titlingpage}
% Folha de rosto
\imprimirfolhaderosto

% Inserir folha de aprovação
\begin{folhadeaprovacao}
	
	\begin{center}
		{\ABNTEXchapterfont\large\imprimirautor}
		
		\vspace*{\fill}\vspace*{\fill}
		\begin{center}
			\ABNTEXchapterfont\bfseries\Large\imprimirtitulo
		\end{center}
		\vspace*{\fill}
		
		\hspace{.45\textwidth}
		\begin{minipage}{.5\textwidth}
			\imprimirpreambulo
		\end{minipage}%
		\vspace*{\fill}
	\end{center}
	
	Banca Examinadora do Trabalho de Conclusão de Curso	
	
	\setlength{\ABNTEXsignwidth}{14cm}
	\assinatura{\textbf{Prof. Dr. Carlos Manuel Dias Viegas - Orientador}} 
	\assinatura{\textbf{Prof. Dr. Danilo Curvelo de Souza}}
	\assinatura{\textbf{Prof. M.Sc. Sérgio Natan Silva}}

	\vspace{1.5cm}
	
	\begin{center}
		\vspace*{0.5cm}
		{\large\imprimirlocal}
		\par
		{\large\imprimirdata}
		\vspace*{1cm}
	\end{center}
	
\end{folhadeaprovacao}

% Dedicatória
\begin{dedicatoria}
	\vspace*{\fill}
	\centering
	\noindent
	\textit{Este trabalho é dedicado a Marta e Osmildo.}
	 \vspace*{\fill}
\end{dedicatoria}

% Agradecimentos
\begin{agradecimentos}

    Este trabalho não poderia ser concluído sem a ajuda de diversas pessoas as
    quais presto minha homenagem:

    Ao meu orientador Professor Dr. Carlos Manuel Dias Viegas, pela paciência,
    confiança e ideias que tornaram possível a elaboração deste trabalho.

    Aos meus pais, Osmildo e Marta, que tanto me guiaram, me incentivaram e
    me apoiaram a crescer espiritualmente e profissionalmente, a minha irmã Catherine e ao meu cunhado Talis,
    que sempre foram um exemplo de perseverança e nunca deixaram de acreditar em mim.

    À minha companheira Juliana por ter me incentivado e me apoiado em todos
    os momentos.

    À toda a minha família e a todos os meus amigos, pois, sem eles, isso não
    seria possível.

\end{agradecimentos}

% ---
% Epígrafe
% ---
\begin{epigrafe}
	\vspace*{\fill}
	\begin{flushright}
		\textit{``Feliz o homem que encontrou a sabedoria e alcançou o entendimento,\\
			porque a sabedoria vale mais do que a prata, \\
			e dá mais lucro que o ouro."\\
			(Bíblia Sagrada, Provérbios 3, 13-14)}
	\end{flushright}
\end{epigrafe}

% RESUMO
% resumo na língua vernácula (obrigatório)
\setlength{\absparsep}{18pt} % ajusta o espaçamento dos parágrafos do resumo
\begin{resumo}

Este trabalho consiste no desenvolvimento de um sistema domiciliar de monitoramento de energia elétrica de baixo custo, em tempo real.
Tem como proposta um dispositivo e um sistema que juntos somam a nova realidade de meios para a economia de energia.
O equipamento tem como base a placa de desenvolvimento NodeMCU e um sensor de corrente que afere o consumo em tempo real de um dispositivo. Através
dos dados coletados e dos cálculos realizados pelo sistema é possível acompanhar o consumo de energia elétrica de uma forma mais fácil.
O intuito é que o constante monitoramento possa trazer uma conscientização da economia de energia.
 
 \noindent
 \textbf{Palavras-chaves}: Economia de energia. Monitoramento. Eficiência Energética. IoT. 
\end{resumo}
% ---
% resumo em inglês
\begin{resumo}[Abstract]
	\begin{otherlanguage*}{english}	
	
	
This work consists of the development of a real-time home monitoring system for electricity, which is fully customizable.
It proposes a device and a system that together add the new reality of means for energy saving.
The equipment is based on the NodeMCU development board and a current sensor that measures the real time consumption of a device. Through
the data collected and the calculations performed by the system, it is possible to monitor the electricity consumption in an easier way.
The intention is that constant monitoring can bring an awareness of energy savings.
	
	\vspace{\onelineskip}
	\noindent 
	\textbf{Keywords}: Energy saving. Monitoring. Energy Efficiency. IoT
	\end{otherlanguage*}
\end{resumo}

% ---
% inserir lista de ilustrações
\pdfbookmark[0]{\listfigurename}{lof}
\listoffigures*
\cleardoublepage

% inserir lista de tabelas
\pdfbookmark[0]{\listtablename}{lot}
\listoftables*
\cleardoublepage

% inserir lista de abreviaturas e siglas
\begin{siglas}
\item[HTML]   \textit{HyperText Markup Language}
\item[CSS]	  \textit{Cascading Style Sheets}
\item[API]    \textit{Application Programming Interface}
\item[HTTP]   \textit{HyperText Transfer Protocol}
\item[TCP]    \textit{Transmission Control Protocol}
\item[SQL]    \textit{Structured Query Language} 	
\item[SDK]    \textit{Software Development Kit}
\item[IoT]    \textit{Internet of Things}
\item[JSON]   \textit{JavaScript Object Notation}
\item[SCT]    \textit{Split-Core Current Transformer}
\item[REST]   \textit{Representational State Transfer}
\item[EPE]    \textit{Empresa de Pesquisa Energética}
\item[ANEEL]  \textit{Agência Nacional de Energia Elétrica}
\item[CEMIG]  \textit{Comapanhia Energética de Minas Gerais}
\item[SIN]    \textit{Sistema Interligado Nacional}
\item[IBGE]    \textit{Instituto Brasileiro de Geografia e Estatística}
\item[SOC]    \textit{System on a Chip}
\end{siglas}

% inserir lista de símbolos
\begin{simbolos}
  \item[$ GWh $] Gigawatt-hora
  \item[$ KWh $] Quilowatt-hora
  \item[$ R\$ $] Moeda corrente oficial da República Federativa do Brasil
\end{simbolos}

% inserir o sumario
\pdfbookmark[0]{\contentsname}{toc}
\tableofcontents*
\cleardoublepage

\textual

% Capitulo 1: Introdução
\chapter[Introdução]{Introdução}
\label{ch:introdução}
A eletricidade se tornou um pilar central na atualidade, sendo uma das principais fontes de força, calor e luz utilizada no  mundo. Entrantando com o
crescente consumo de energia elétrica nos últimos tempos, a demanda por produção da mesma teve um crecismento significativo, trazendo consigo 
impactos ambientais e econômicos. O Brasil por mais que possua em seu território grandes possibilidade para a construção de hidrelétricas, não
está isento do problema da alta demanda por energia elétrica. Problema que se agravou em 2015 quando o país começou a passar por
uma crise hídrica.

Como a \autoref{fig:rede_convencional} mostra, a maior parte da energia elétrica gerada no Brasil é por meio de hidroelétricas, essa dependência
energetica junto com a crise hídrica que o país sofreu cuminou em uma política de racionamento e aumento dos impostos - taxa inflacionária no
consumo de energia elétrica - que impactou diretamente a vida de cada cidadão brasileiro, trouxe consequências, como o aumento do 
custo da energia elétrica. Segundos dados (G1, 2016) entre 2015 e 2016 a crise hídrica no Brasil não interferiu apenas na conta de luz mas trouxe
um aumento na inflação do país.

% sustentabilidade


% JONATHAN aqui seu capítulo introdutório. Ele pode conter figuras, tabelas e subseções. Exemplo de uma citação indireta \cite{yu2011new}, e da \autoref{fig:rede_convencional}. Imagens do autor, tem na fonte o texto "Elaborado pelo autor".

\begin{figure}[h!]
	\includegraphics[width=0.9\textwidth, keepaspectratio=true]{forma_energia}
	\centering
	\caption[Capacidade de instalada de geração elétrica no Brasil (MW)]{Capacidade de instalada de geração elétrica no Brasil (MW)}
	\fonte{\cite[p. 15]{cgee}.}
	\label{fig:rede_convencional}
\end{figure}
\FloatBarrier

A necessidade de contornar os desafios da crescente demanda energetica insentiva a busca por fontes alternativas e limpas de energia. O Brasil 
possui 42,30\% de fontes renováveis da sua matriz energetica e esse número deve aumentar até 2021 onde alcançará a marca de 85,00\% (as hidroelétricas
estão inclusas nesse meio), segundo o Ministério de Minas e Energia. No Plano Decemal de Expansão de Energia (PDE) 2020, o gorveno brasileiro
assume que a sustentabilidade é a chave mestra para a expenssão de atividades de geração de energia elétrica. A \autoref{fig:fonte_energia} mostra
que o Brasil vem investindo ao passar dos anos em fontes limpas de energias. Contudo também mostra que o Brasil ainda é muito dependente das 
hidrelétricas que apesar de ser uma fonte limpa e renovável traz malefícios como as grandes áreas alagadas em volta da represa, impactando no
ciclo de vida das espécies e obriga populaões ribeirinhas a migrarem, isso mostra que não basta apenas ter fontes limpas
e renováveis de energia, é necessário buscar melhorias como as Smart Grids e técnicas como de Smart metering.




\begin{figure}[h!]
	\includegraphics[width=0.7\textwidth, keepaspectratio=true]{fontes_energia}
	\centering
	\caption[Fontes de geração de energia elétrica (GWh)]{Fontes de geração de energia elétrica (GWh)}
	\fonte{\cite[p. 15]{cgee}.}
	\label{fig:fonte_energia}
\end{figure}
\FloatBarrier

Exemplo de citação direta com menos de três linhas. Como "o mercado de energia elétrica está baseado em tarifas fixas e limitações de informações em tempo real sobre gerenciamento da rede e da carga" \cite[p. 15]{cgee}, o consumidor acaba, então, não tendo como optar por fornecimentos elétricos mais adequados. 


\section{Uma subseção explicativa}

Lorem ipsum, uma citação direta 

\begin{citacao}[brazil]
[...] redes elétricas que podem, de forma inteligente, integrar o comportamento e as ações de todos os usuários conectados a ela, como geradores, consumidores e os que desempenham as duas funções, para entregar, eficientemente, um fornecimento de eletricidade sustentável, econômico e seguro \cite[p. 51, tradução livre]{yu2011new}.
\end{citacao}

Para compreender melhor as grandes mudanças e os benefícios gerados pelas \textit{Smart Grids} no contexto do fornecimento elétrico, a \autoref{tab-comparativa} traz um breve comparativo entre as redes tradicionais e as redes inteligentes.

\begin{table}[!ht]
\centering
\resizebox{\textwidth}{!}{%
\begin{tabular}{ll}
\hline
\multicolumn{1}{c}{\textbf{Redes Elétricas Tradicionais}} & \multicolumn{1}{c}{\textbf{Redes Elétricas Inteligentes}}                 \\ \hline
\rowcolor[HTML]{DDDDDD} 
Eletromecânica, estado sólido                             & Digital/Microprocessadores                                                \\
Unidirecional e localmente bidirecional                   & Global/comunicação bidirecional integrada                                 \\
\rowcolor[HTML]{DDDDDD} 
Geração centralizada                                      & Acomoda geração distribuída                                               \\
{Controle, monitoramento e proteção limitados}  & WAMPAC, proteção adaptativa \\
\rowcolor[HTML]{DDDDDD} 
"Cega"                                                    & Auto-monitoramento                                                        \\
Recuperação manual                                        & Auto-reconfigurável                                                       \\
\rowcolor[HTML]{DDDDDD} 
Checagem manual de equipamentos                           & Monitoração remota de equipamentos                                        \\
Sistema de controle de contingências limitado             & Sistema de controle pervasivo                                             \\
\rowcolor[HTML]{DDDDDD} 
Confiabilidade estimada                                   & Confiabilidade preditiva                                                 
\end{tabular}%
}
\caption{Comparação entre redes elétricas convencionais e redes elétricas inteligentes}
\label{tab-comparativa}
\fonte{\cite[p. 28, tradução nossa]{ali2013smart}}
\end{table}

\section{Trabalhos Relacionados}
\lipsum[1-1]

\section{Motivação}
O que lhe motiva a realizar este trabalho.

\section{Objetivos}
Objetivo geral e específicos.

\section{Estrutura do Trabalho}
Este trabalho apresenta uma introdução sobre o tema, mostrando os fatores que motivam a implantação da ideia, além da justificativa e dos objetivos. Em sequência, o \autoref{ch:cap2} aborda (...). O \autoref{ch:cap3}, por sua vez, explica a metodologia para ..., enquanto o \autoref{ch:cap4} trata de (...). O \autoref{ch:cap5} apresenta (...). Por fim, o \autoref{ch:cap6} traz as principais conclusões e contribuições deste trabalho.

% Capitulo 2
\chapter[Embasamento Teórico]{Embasamento Teórico}
\label{ch:cap2}
\section[\textit{Contextualizacao}]{\textit{Contextualização}}\label{context}
O Consumo de energia elétrica é um dos principais indicadores de desenvolvimento e de qualidade de vida 
de um país. Esse índice é tão importate que reflete diretamente no rítimo de vida de uma populção, pois mostra
se as atividades industriais de uma nação está ou não em um bom rítimo e pode detectar se o comércio está em alta,
devido aos bens e serviços que o povo adiquiriu. Porém um crescimento desordenado na população e um crescimento
exponencial no consumo de enérgia pode acarretar em problemas para um determinado país.
Analisando os dados \cite{epe-balanco-final}, o consumo de energia
elétrica no Brasil vem crescendo ao longo dos anos, o brasileiro vem consumindo mais energia elétrica, nos últimos
35 anos teve um crescimento médio de 6,72\% dessa demanda, após a crise que o Brasil sofreu entre os anos 2002 e 2005 houve um crescimento
de 4,91\% na demanda energética do país. A \autoref{fig:consumo_energia_total} nos mostra bem o cenário de crise energética que o Brasil vinha
passando ao longo dos anos, até 2008 o país consumia mais do que produzia.

\begin{figure}[h!]
	\includegraphics[width=0.85\textwidth, keepaspectratio=true]{consumo_energia_total}
	\centering
	\caption[Estrutura do Consumo de fontes primárias]{Estrutura do Consumo de fontes primárias}
	\fonte{\cite[p. 43]{epe-balanco-final}.}
	\label{fig:consumo_energia_total}
\end{figure}
\FloatBarrier

O governo brasileiro tomou algumas medidas estratégicas para poder acompanhar a crescente demanda por energia elétrica, constituiu o planejamento
da construção de mais de 80 usinas até 2020, hidroelétricas, temoelétricas e até usina nuclear. Um grande problema desse planjamento que o gorverno
fez são os inumeros impactos ambientais e econômicos, um exemplo prático é a usina de de Belo Monte - Rio Xingu, Pará - obra que foi planejada
para ser a quarta maior hidroelétrica do mundo, a maior do Brasil, com capacidade habastecer 40\% das recidências, foi orçada em R\$ 30 bilhões
deveria ter seu início de operação no segundo semestre de 2015 mas até os dias atuais não entrou em funcionamento. Vale salientar que a construção
trouxe o desmatamento de áreas indígenas, alagamentos permanentes, comprometimento da fauna e flora e aumento da dificuldade dos transportes fluviais
de comunidades ribeirinhas.

Analisando a grande demanda energetica que o brasileiro vem requerindo e levando em conta as consquências negativas do planejamento das 80 usinas,
surgi uma questão bastante recorrente: - "O que fazer? Constuir usinas mesmo sabendo dos impactos negativos que podem surgir, ou não construí-las e
aumentar a tarifação pelo consumo de energia visando diminuir o consumo?" - A resposta para essas e outras questões que podem aparecer não são fáceis.
Entretanto o governo brasileiro optou por deixar o consumo de energia elétrica mais caro, principalmente nos horários de pico. A evolução da tarifa,
pode ser observada na \autoref{evolucao-tarifa}


\begin{table}[!ht]
	\centering
	\begin{tabular}{lcccc}
	\hline
	\textbf{Ano} & \multicolumn{1}{l}{\textbf{1º Trimestre}} & \multicolumn{1}{l}{\textbf{2º Trimestre}} & \multicolumn{1}{l}{\textbf{3º Trimestre}} & \multicolumn{1}{l}{\textbf{4º Trimestre}} \\ \hline
	\rowcolor[HTML]{DDDDDD} 
	2013         & 120,8                                     & 117,1                                     & 114,5                                     & 116,1                                     \\
	2014         & 121,1                                     & 127,6                                     & 134,4                                     & 141,9                                     \\
	\rowcolor[HTML]{DDDDDD} 
	2015         & 154,2                                     & -                                         & -                                         & -                                        
	\end{tabular}
	\caption{Evoluçao dos custos de energia elétrica em R\$/MWh}
	\fonte{\cite[p. 1]{evolucao-tarifa-ref}}
	\label{evolucao-tarifa}
\end{table}

Uma medida totalmente cabível que ainda é desconhecida por alguns brasileiros é a chamada "\textit{exposição da informação}", deixando sempre bem claro 
quanto o consumidor tem gastado ou consumindo ao longo do mês em sua recidência, isso é possível graças a equipamentos que estão sempre monitorando
a rede elétrica.Segundo uma pesquisa realizada pela Associação Brasileira das Empresas de Serviços de Conservação de Energia, seis anos o Brasil 
desperdiçou o equivalente a 250GWh em energia o que equivale a R\$62 bilhões, desperdíciu que se deu justamente a tamanha falta de infomação que 
o consumidor tem, se ao saber o quanto tem consumido ou gastado em tempo real o consumidor poderia se prevenir dos desperdícios. 

\subsection[\textit{Setor Energético Brasileiro}]{\textit{Setor Energético Brasileiro}}\label{seb}
Ao passar dos anos o Brasil vem mostrando cada vez mais o seu potencial na produão de energia, o território brasileiro possibilita as várias formas
de obtenção da eletricidade. Analisando os dados \cite[p.29]{epe-anuario-2015} e comparando com a \autoref{cap_ele} nota-se que o Brasil subiu duas
posoções no \textit{Rank} de geração de energia elétrica, isso é reflexo do aumentou da capicidade de produção de energia que chegou na marca de 8,39\%.

\begin{table}[!ht]
	\centering
	\begin{tabular}{lccccc}
		\rowcolor[HTML]{9B9B9B} 
		\multicolumn{1}{c}{\cellcolor[HTML]{9B9B9B}} & {\color[HTML]{FFFFFF} \textbf{2010}} & {\color[HTML]{FFFFFF} \textbf{2011}} & {\color[HTML]{FFFFFF} \textbf{2012}} & {\color[HTML]{FFFFFF} \textbf{2013}} & \multicolumn{1}{l}{\cellcolor[HTML]{9B9B9B}{\color[HTML]{FFFFFF} \textbf{2014}}} \\ \hline
		\textbf{Mundo}                               & \multicolumn{1}{l}{\textbf{5080,6}}  & \multicolumn{1}{l}{\textbf{5305,0}}  & \multicolumn{1}{l}{\textbf{5514,6}}  & \multicolumn{1}{l}{\textbf{5736,2}}  & \multicolumn{1}{l}{\textbf{6038,7}}                                              \\ \hline
		\rowcolor[HTML]{DDDDDD} 
		China                                        & 971,8                                & 1069,5                               & 1154,6                               & 1267,7                               & 1399,5                                                                           \\
		Estados Unidos                               & 1039,1                               & 1051,3                               & 1063,0                               & 1060,1                               & 1074,6                                                                           \\
		\rowcolor[HTML]{DDDDDD} 
		Japão                                        & 284,9                                & 287,3                                & 293,3                                & 300,8                                & 313,4                                                                            \\
		Índia                                        & 213,1                                & 246,0                                & 260,3                                & 283,0                                & 310,8                                                                            \\
		\rowcolor[HTML]{DDDDDD} 
		Rússia                                       & 228,1                                & 231,6                                & 233,6                                & 235,2                                & 247,6                                                                            \\
		Alemanha                                     & 162,7                                & 167,5                                & 177,3                                & 186,1                                & 198,4                                                                            \\
		\rowcolor[HTML]{DDDDDD} 
		Canadá                                       & 132,3                                & 132,9                                & 130,7                                & 133,3                                & 136,8                                                                            \\
		Brasil                                       & 11,3                                 & 117,1                                & 121,0                                & 126,7                                & 133,9                                                                           
	\end{tabular}
	\caption{Capacidade instalada de geração elétrica no mundo, 2014 (GW)}
	\fonte{\cite[p. 29]{epe-anuario}}
	\label{cap_ele}
\end{table}

A maior produção de energia do Brasil provem das hidroelétricas, o país é referência mundial quando o assunto é obtenção de energia através de
usinas hidroelétricas - \autoref{cap_hidro} - isso é possível devido a sua alta concentração de rios de grande porte e ao grande volume de chuva
que alimenta e reforça o poderio hídrico do país. A energia que a usina hidroelétrica fornece é conseguida através da energia hidráulica que provém
do aproveitamento da força potencial e cinética das correntes de água,rio, mar. A água ao passar por tubulações com muita força e velociadade 
movimentas as turbinas fazendo com que elas girem em um velociadade suficiente para que os geradores acoplados nas turbinas, transformem energia
mecânica em energia elétrica, lembrando que a eficiência energética de uma usina hidroelétrica é de 65,2\%. Após esse longo processo a energia 
extraída é enviada para estações de tratamento e após essa etapa é enviada para a matriz energetica que fará a distribuião da energia extraída. 

\begin{table}[!ht]
	\centering
	\begin{tabular}{lccccl}
		\rowcolor[HTML]{9B9B9B} 
		\multicolumn{1}{c}{\cellcolor[HTML]{9B9B9B}} & {\color[HTML]{FFFFFF} \textbf{2010}} & {\color[HTML]{FFFFFF} \textbf{2011}} & {\color[HTML]{FFFFFF} \textbf{2012}} & {\color[HTML]{FFFFFF} \textbf{2013}} & {\color[HTML]{FFFFFF} \textbf{2014}} \\ \hline
		\textbf{Mundo}                               & \multicolumn{1}{l}{\textbf{903,9}}   & \multicolumn{1}{l}{\textbf{929,9}}   & \multicolumn{1}{l}{\textbf{957,5}}   & \multicolumn{1}{l}{\textbf{1000,4}}  & \textbf{1038,3}                      \\ \hline
		\rowcolor[HTML]{DDDDDD} 
		China                                        & 199,5                                & 214,6                                & 229,1                                & 258,9                                & 283,0                                \\
		Brasil                                       & 80,7                                 & 82,5                                 & 84,3                                 & 86,0                                 & 89,2                                 \\
		\rowcolor[HTML]{DDDDDD} 
		Estados Unidos                               & 78,8                                 & 78,7                                 & 78,7                                 & 79,2                                 & 79,7                                 \\
		Canadá                                       & \multicolumn{1}{l}{74,9}             & \multicolumn{1}{l}{75,4}             & \multicolumn{1}{l}{75,4}             & \multicolumn{1}{l}{75,4}             & 75,4                                
	\end{tabular}
	\caption{Capacidade instalada de geração hidrelétrica no mundo, 2014 (GW)}
	\fonte{\cite[p. 30]{epe-anuario}}
	\label{cap_hidro}
\end{table}

É do conhecimento de qualquer brasileiro que possua uma noção básica de geografia que a região norte é a região que possui a maior quantidade de rios,
essa noção pode levar uma conclusão errada - A região norte é a que mais produz energia - pois nem todo rio tem potencial para que uma hidroelétrica se instale.
Por sua vez as regiões sul e suldeste são as que mais necessitam de energia, devido a densidade populacional e a quantidade de insdutrias instaladas nas regiões.
A \autoref{pxcxge} externa essa problemática de uma manéira bem visível. Perceb-se que por exemplo a região sudeste é a que produz mais energia, porém é a que mais
gasta, sendo os gastos maiores do que os ganhos, já a região norte e nordeste são regiões que produzem mais do que gastam. Vendo esse total desequilíbrio
de geração e consumo de energia, surgiu a necessidade da criação do Sistema Interligado Nacional (SIN). O SIN é constituido por todas as regiões brasileiras
e é interconectado por meio de uma malha de trasmissão que propicia a transferência de energia entres os subsistemas, permitindo a obtenção de ganhos
sinérgicos e explora a diversidade entre os regimes hidrológicos e das bacias. A integração dos recursos de geração e transmissão permite o atendimento ao
mercado com segurança e economicidade.
\begin{table}[!ht]
	\centering
	\begin{tabular}{cccc}
		\hline
	\textbf{Região} & \textbf{População} & \textbf{Consumo em GW} & \textbf{\begin{tabular}[c]{@{}c@{}}Capacidade Instalada de \\ Geração Elétrica GW\end{tabular}} \\ \hline
		\rowcolor[HTML]{DDDDDD} 
		Norte           & 17.707.783         & 12.197                 & 25,484                                                                                          \\
		Nordeste        & 56.915.936         & 12.109                 & 29,803                                                                                          \\
		\rowcolor[HTML]{DDDDDD} 
		Sudeste         & 86.356.952         & 74.584                 & 44,810                                                                                          \\
		Sul             & 29.439.773         & 19.173                 & 31,681                                                                                          \\
		\rowcolor[HTML]{DDDDDD} 
		Centro-Oeste    & 15.660.988         & 5.634                  & 18,558                                                                                         
	\end{tabular}
	\caption{Relação População x Consumo por Região x Geração Elétrica por Região}
	\fonte{(IBGE e EPE)}
	\label{pxcxge}
\end{table}


\subsection[\textit{Medição de Energia}]{\textit{Medição de Energia}}\label{med-energia}

Após enterder todo o funcionamento da geração e distribuição de energia no Brasil, é conveniente entender o processo de leitura do consumo de 
energia elétrica, assim como as questões que esse trabalho faz a respeito da eficácia. Tendo a possibilidade de atualizar esse sistema com novas 
tecnologias que proporcinam maior segurança e menores custos ao consumidor.

Os primeiros medidores de eletricidade foram utilizados na operação de lâmpadas em série, um vez que a tensão era constante, a corrente exigida
por cada lâmpada era conhecida e todas estavam ligadas no mesmo interruptor, os medidores foram suficientes apenas para medir o gasto das lâmpadas
em um tempo determinado, surgindo o termo - lâmpada-hora. Em 1872 o pesquisador Samuel Gardiner trouxe a toda a primeira patente sobre um contator 
de energia, que era formado por uma lâmpada acoplada a um contador de energia DC controlado por um relógio e um eletroímã, ao passar do tempo várias
outras patentes foram surgindo e tentando melhor o projeto de Samuel Gardiner, mas foi apenas em 1892 que que surgiu o primeiro medidor de watt-hora
com precisão e confiabilidade suficiente para aplicação em medição de consumo de energia. Criado por Thomas Duncan, inicialmente seu objetivo era a medição
de circuitos monofásicos, porém com o bom desempenho do aprelho modificações foram feitas para à medição de circuitos polifásicos de energia.

Atualmente a energia elétrica é quantificada através de um equipamento chamado medidor, que nos dias atuais a medição é feita em quilowatt-hora.
Os medidores da atualidade são caracterizados por padrões da norma NBR 14519, o grupo de medidor mais utilizado pelas concessionárias nas residências
é o grupo B. 

\begin{itemize}
	\item Grupo B \\
	É caracterizado por unidades consumidoras de baixa tensão, com tensões inferiores a 2,3KV. As unidades consumidores podem ser classificadas
	mediante a necessidade da concessionária responsável, geralmente o tipo B1 é residencial, tipo B2 são as residências rurais e estabelecimentos
	comerciais ou insdustriais são classificados como o tipo B3.
\end{itemize}

Estima-se que 92\% dos medidores em funcionamento são eletromecânicos, pois são de baixa custo e de boa qualidade, com o erro máximo de 2\% de seu valor
nominal de operação. Não ter um medidor em uma unidade consumidora pode gerar transtornos tanto para concessionário, pois não saberá o quanto deve cobrar ao 
consumidor, como para o dono do estabelecimento, pois não terá o aporte devido prestado pela concessionária de energia.



% Capitulo 3
\chapter[Desenvolvimento]{Desenvolvimento}
\label{ch:desenvolvimento}
O \textit{Power Monitor} surgiu da necessidade da conscientização do gasto energético e da melhor compreensão da conta de luz. Baseado nesse conceito,
foi desenvolvido um \textit{software} que permite uma fácil comunicação com qualquer equipamento construído que tenha a finalidade de monitorar a energia elétrica e um \textit{hardware} para demonstração
da comunicação entre ambos. O sistema traz uma forma mais fácil e próxima do consumidor final de se quantificar a energia elétrica consumida em um estabelecimento. No lugar do Quilowatt-hora, medida que é usada atualmente,
o \textit{software} propõe mensurar o gasto energético em reais (R\$), trazendo a realidade do consumo mensal para mais próximo de cada brasileiro.

Nesse capítulo será mostrado todo o passo a passo para o desenvolvimento do \textit{software} e \textit{hardware}, juntamente com a comunicação 
entre ambos, por fim será mostrado os resultados obtidos. 


\section[\textit{Visão Geral}]{\textit{Visão Geral}}\label{visal-geral}

Em resumo pode-se ter uma visão geral de como o ambiente - \textit{software} e \textit{hardware} - funciona observando a \autoref{fig:diagrama-vg}.
O sistema \textit{web} é responsável por fazer a comunicação entre o banco de dados e os dispositivos, já os eletrodomesticos são gerenciados
pelo ESP8266 que possui uma comunicação direta via \textit{websocket} com o sistema \textit{web}.

\begin{figure}[h!]
	\includegraphics[width=0.7\textwidth, keepaspectratio=true]{diagrama-1}
	\centering
	\caption[Visão geral do ambiente]{Visão geral do ambiente}
	\label{fig:diagrama-vg}
\end{figure}
\FloatBarrier

\section[\textit{Software}]{\textit{Software}}\label{soft-sec}
O controle dos dispositivos de um cômodo, que estão interligados com o \textit{ESP8266} são controlados pelo \textit{software}. Dessa forma
todos os dispositivos que possuem comunicação com o microcontrolador e que estão cadastrados nos sistema podem ser controlados (Ligar/Desligar) e também
é possível ter um acompanhamento dos gastos.

O sistema possui uma interface \textit{web} que pode ser acessada por qualquer dispositivo que tenha acesso a internet e possua um 
\textit{browser}. O \textit{software} possui uma interface de apenas um único usuário, ao acessar o sistema o usuário se depara com um
visual bem agradável e fácil de se usar. Ao entrar no sistema o usuário visualiza a página principal, \autoref{fig:principal-ft}, nela encontram-se
as principais informações que o usuário irá precisar, como também mostrar as oções de cadastrar um novo dispositivo, listar os dispositivos, cadastrar um novo cômodo,
listar um novo cômodo etc.

\begin{figure}[h!]
	\includegraphics[width=1.0\textwidth, keepaspectratio=true]{principal}
	\centering
	\caption[Tela inicial do sistema]{Tela inicial do sistema}
	\label{fig:principal-ft}
\end{figure}
\FloatBarrier

Todas as informações colhidas pelo servidor em \textit{node.js} (\autoref{node}), são recebidas e tratadas pelo sistema \textit{web}. Os dados
são importantíssimos, pois mediante eles é que se torna possível a contrução dos gráficos e das previsões fornecidas pelo sistema. No \textit{Power Monitor}
a forma de comunicação com o banco de dados é feita mediante as chamadas de API, existe uma chamada para cada ação prevista no sistema. A \autoref{fig:api-ft}
retrata bem esse cenário, pode-se perceber que o \textit{end-point} (expressão utilizada para se referenciar a um extremidade de um canal de comunicação, portanto, isso 
seria representado como a URL de um servidor ou serviço.) \textbf{get-comodos} é destinado a obtenção de todos os cômodos cadastrados já o \textit{end-point}
\textbf{get-dispositivos} é destinado a obtenção de todos os dispostivos cadastrados. O motivo da comunicação entre servidor \textit{web} e sistema \textit{web}
ser feita via chamada de API é bem simples, pois qualquer sistema seja \textit{web, desktop} ou qualquer outro tipo, basicamente precisa ter uma comunicação com a internet
para utilizar o \textit{Power Monitor}. O \textit{software web} não precisa obrigatoriamente de internet para poder funcionar, pois a comunicação entre servidor e sistema é
baseada em uma rede local, justamente para que o \textit{software} não dependa de terceiros. Para uma perfeita comunicação o ambiente só precisa está configurado
na mesma rede \textit{Wi-Fi}. 

\begin{figure}[h!]
	\includegraphics[width=1.0\textwidth, keepaspectratio=true]{api}
	\centering
	\caption[Chamada de API]{Chamada de API}
	\label{fig:api-ft}
\end{figure}
\FloatBarrier

O \textit{software} pode ser dividido em duas partes, servidor \textit{web} e interface \textit{web}. O servidor foi desenvolvido usando a linguagem
\textit{javascript} (\autoref{js}) e para auxílio foi utilizado o \textit{framework node.js} (\autoref{node}), a comunicação entre servidor e banco de dados
é feita pelo \textit{MySQL Server} (\autoref{sql}). A interface \textit{web} é o agente consumidor de todos esses serviços, com uma comunicação via 
\textit{webscoket} (\autoref{websocket}) com o servidor é capaz de receber e enviar dados a qualquer instante. A combinação desses três serviços - servidor \textit{web},
servidor do banco de dados e interface \textit{web} - resultou em uma aplicação intuitiva e amigável, desenvolvida para dar o total suporte às análises do dados
enviados pelo \textit{hardware}. A \autoref{fig:banco-dados} representa o esquema das tabelas do banco de dados que foi utulizado no \textit{Power Monitor}.

Como já foi apresentada a \autoref{fig:principal-ft} representa a tela inicial da interface \textit{web}, vê-se que é possível
visualizar o total gasto no mês decorrente, o progresso dos gastos por cômodo que por sua vez é baseado na espectativa de gasto mensal (\autoref{fig:configuracao-ft}), a lista
de todos os cômodos cadastrados e um gráfico mostrando o consumo por cômodo referente ao mês atual.

\begin{figure}[h!]
	\includegraphics[width=1.0\textwidth, keepaspectratio=true]{banco}
	\centering
	\caption[Esquema das tabelas do banco de dados]{Esquema das tabelas do banco de dados}
	\label{fig:banco-dados}
\end{figure}
\FloatBarrier

As figuras: \ref{fig:c-comodo}, \ref{fig:l-comodo}, \ref{fig:c-dispositivo}, \ref{fig:l-dispositivo} e \ref{fig:configuracao-ft}, 
são relacionadas as telas de cadastro, listagem e de configuração da interface \textit{web}. Nelas é possível cadastrar e listar comodôs e dispositivos assim como 
configurar alguns parâmetros do sistema como o preço da trarifa cobrado por KWh pela empresa resposável e a espectativa de gasto mensal.

Uma vez que os cômodos, dispositivos e parâmetros são cadastrados no sistema é possível gerencia-los através da edição ou exclusão dos dados, que se torna possível
nas telas de listagem, para os cômodos e dispositivos, já os parâmetros do sistema na própria tela de configuração se faz a edição dos dados. 

\begin{figure}[h!]
	\includegraphics[width=1.0\textwidth, keepaspectratio=true]{c-comodo}
	\centering
	\caption[Cadastro de um cômodo]{Cadastro de um cômodo}
	\label{fig:c-comodo}
\end{figure}
\FloatBarrier

\begin{figure}[h!]
	\includegraphics[width=1.0\textwidth, keepaspectratio=true]{l-comodo}
	\centering
	\caption[Lista dos cômodos cadastrados]{Lista dos cômodos cadastrados}
	\label{fig:l-comodo}
\end{figure}
\FloatBarrier

\begin{figure}[h!]
	\includegraphics[width=1.0\textwidth, keepaspectratio=true]{c-dispositivo}
	\centering
	\caption[Cadastro de um dispositivo]{Cadastro de um dispositvio}
	\label{fig:c-dispositivo} 
\end{figure}
\FloatBarrier

\begin{figure}[h!]
	\includegraphics[width=1.0\textwidth, keepaspectratio=true]{l-dispositivo}
	\centering
	\caption[Listagem dos dispositivos cadastrados]{Listagem dos dispositivos cadastrados}
	\label{fig:l-dispositivo} 
\end{figure}
\FloatBarrier

\begin{figure}[h!]
	\includegraphics[width=1.0\textwidth, keepaspectratio=true]{configuracao}
	\centering
	\caption[Tela de configuração dos parâmetros do sistema]{Tela de configuração dos parâmetros do sistema}
	\label{fig:configuracao-ft} 
\end{figure}
\FloatBarrier

Uma vez que todos os cômodos, dispositivos e parâmetros já se encontram cadastrados e já exista a comunicação establecida com o hardware (***), a interface
\textit{web} disponibiliza um séria de formas para visualização das informações coletadas e tratadas. Na \autoref{fig:extrato} é possível visualizar
uma espécie de extrato do consumo de todos os dispositivos, podendo perceber quando foram ligados, quando foram desligados e assim resultando no total gasto.
Já as figuras: \ref{fig:ano-c}, \ref{fig:matu-c} e \ref{fig:mant-c} referem-se aos gráficos que são contruídos baseados nas informações coletadas pelo sistema.

Os gráficos por sua vez são construídos mediante aos calculos que o sistema faz usando como base a \autoref{eq-consumo}, o resusltado dessa conta
fornece ao sistema o consumo em reais (R\$) do dispostivo em um dado intervalo de tempo conhecido. Vale salientar que o resultado é o esperado já que o consumo
do dispositivo (KWh) é pré definido pelo usuário (\autoref{fig:c-dispositivo}), para o cálculo real do consumo usa-se a \autoref{eq-potencia} que leva em conta a corrente real que passa
pelo dispositivo ao longo do tempo que ele permanece ligado. Com esses dados é que se torna possível a construção dos gráficos e extrado presentes no sistema.

\begin{equation} \label{eq-consumo}
	\frac{consumo \, \, do \, \, dispositivo \times tempo \, \, de \, \, uso}{número \, \, de \, \, dias \, \, no \, \, mês} \times tarifa
\end{equation}

\begin{equation} \label{eq-potencia}
	 \frac{ (tensão \times corrente) \times horas \, \, de \, \, uso \, \, por dia \times número \, \, de \, \, dias \, \, no \, \, mês}{1000}
\end{equation} 

\begin{figure}[h!]
	\includegraphics[width=1.0\textwidth, keepaspectratio=true]{extrato}
	\centering
	\caption[Demonstrativo do gasto de cada dispositivo]{Demonstrativo do gasto de cada dispositivo}
	\label{fig:extrato} 
\end{figure}
\FloatBarrier

\begin{figure}[h!]
	\includegraphics[width=1.0\textwidth, keepaspectratio=true]{ano-c}
	\centering
	\caption[Consumo geral de todos os dispositivos por cômodo ao longo do ano]{Consumo geral de todos os dispositivos por cômodo ao longo do ano}
	\label{fig:ano-c} 
\end{figure}
\FloatBarrier

\begin{figure}[h!]
	\includegraphics[width=1.0\textwidth, keepaspectratio=true]{matu-c}
	\centering
	\caption[Consumo geral de todos os dispositivos por cômodo ao longo do mês atual]{Consumo geral de todos os dispositivos por cômodo ao longo do mês atual}
	\label{fig:matu-c} 
\end{figure}
\FloatBarrier

\begin{figure}[h!]
	\includegraphics[width=1.0\textwidth, keepaspectratio=true]{mant-c}
	\centering
	\caption[Consumo geral de todos os dispositivos por cômodo ao longo do mês anterior]{Consumo geral de todos os dispositivos por cômodo ao longo do mês anterior}
	\label{fig:mant-c} 
\end{figure}
\FloatBarrier

O \textit{softaware} também possui um local específico para a gerência dos cômodos, um espaço destinado para a vinculação de dispositivos ao cômodo,
visualização do total gasto em relação ao mês atual em forma de gráfico e a opção de ligar, desligar ou excluir o dispostivo do cômodo. As figuras:
\ref{fig:comodo-ft}, \ref{fig:v-dispositivo} e \ref{fig:a-dispositivo} retratam o cenário descrito.

\begin{figure}[h!]
	\includegraphics[width=1.0\textwidth, keepaspectratio=true]{comodo}
	\centering
	\caption[Visão geral da gerência de um cômodo]{Visão geral da gerência de um cômodo}
	\label{fig:comodo-ft} 
\end{figure}
\FloatBarrier

\begin{figure}[h!]
	\includegraphics[width=1.0\textwidth, keepaspectratio=true]{v-dispositivo}
	\centering
	\caption[Vincular dispositivo ao cômodo]{Vincular dispositivo ao cômodo}
	\label{fig:v-dispositivo} 
\end{figure}
\FloatBarrier

\begin{figure}[h!]
	\includegraphics[width=1.0\textwidth, keepaspectratio=true]{a-dispositivo}
	\centering
	\caption[Ações que podem ser realizadas no dispositivo vinculado]{Ações que podem ser realizadas no dispositivo vinculado}
	\label{fig:a-dispositivo} 
\end{figure}
\FloatBarrier

Outras duas funcionalidades do sistem são os alarmes que são gerados quando o usuário passa dos 50\% e dos 80\% do consumo esperado, cadastrado previamente pelo mesmo e
também o simulador de gastos, onde é possível calcular o gasto mensal e diário que um dispositivo irá trazer para uma residência. As figuras
\ref{fig:aviso-a}, \ref{fig:aviso-v} e \ref{fig:simulador} retratam o descrito.

\begin{figure}[h!]
	\includegraphics[width=1.0\textwidth, keepaspectratio=true]{aviso-a}
	\centering
	\caption[Alerta exibido quando o consumo supera os 50\% do previsto]{Alerta exibido quando o consumo supera os 50\% do previsto}
	\label{fig:aviso-a} 
\end{figure}
\FloatBarrier

\begin{figure}[h!]
	\includegraphics[width=1.0\textwidth, keepaspectratio=true]{aviso-v}
	\centering
	\caption[Alerta exibido quando o consumo supera os 80\% do previsto]{Alerta exibido quando o consumo supera os 80\% do previsto}
	\label{fig:aviso-v} 
\end{figure}
\FloatBarrier

\begin{figure}[h!]
	\includegraphics[width=1.0\textwidth, keepaspectratio=true]{simulador}
	\centering
	\caption[Simulador de gastos]{Simulador de gastos}
	\label{fig:simulador} 
\end{figure}
\FloatBarrier

\section[\textit{Hardware}]{\textit{Hardware}}\label{hard-sec}
O desenvolvimento do \textit{hardware} para demonstação da comunicação com o \textit{power monitor} envolve uma série de sensores e componentes
eletrônicos. A plataforma de prototipagem eletrônica utilizada para a construção desse \textit{hardware}, foi o ESP8266(\autoref{esp}). O principal sensor
utilizado foi o SCT 013-000 (\autoref{sct}), que tem o papel de aferir dados da corrente que passa pelos dispositivos ao longo do tempo que o mesmo se encontra
ligado, também vale destacar o uso do relé que é responsável por toda a lógica de liga e desliga do dispositivo. O ESP8266 faz o intermédio da comunicação entre 
\textit{hardware} e servidor \textit{web}, fazendo toda a comunicação eletrônica entre o sensor e o circuito montado (**), e enviando os dados
recebidos pelo sensor de corrente para o banco de dados via comunicação \textit{websocket} com o servidor \textit{web}. Com relação aos dados enviados 
para o banco de dados, foi feito um algoritmo que quando identifica que o dispositivo está ligado fica verificando a corrente média que 
passa pelo dispostivio e quando o mesmo é desligado é calculada uma média das correntes, esse resultado é multiplicado pelo valor da tensão
e assim é obtido o consumo (KWh) do dispositivo. Após esse processo a informação é levada ao banco de dados e consumida pela interface \textit{web}.

É válido lembrar que, para a comunicação \textit{websocket} é necessário o \textit{hardware} e o \textit{softaware} estarem conectados
na mesma rede \textit{Wi-Fi}. O grande motivo para a escolha do ESP8266 como plataforma de prototipagem foi a sua fácil comunicação com uma rede
\textit{Wi-Fi}, o código a seguir é um exemplo de como estabelecer a comunicação com uma rede sem fio.

\newpage

\begin{lstlisting}
	#include <ESP8266WiFi.h>
	const char* ssid = NOME DA REDE;
	const char* password = SENHA;
	
	WiFi.begin(ssid, password);
\end{lstlisting}

Após estabelecer a conecção o próximo passo será interligar o servidor \textit{web} com o \textit{hardware} através da comunicação por \textit{websocket},
que será facilitada por meio da biblioteca \textit{SocketIOClient}, ela fornece alguns métodos como: \textit{\textbf{emit}}\protect\footnotemark, \textit{\textbf{on}}\protect\footnotemark 
e \textit{\textbf{connect}}\protect\footnotemark  que ajudam no momento de concretizar a comunicação total do \textit{hardware}. A seguir terá um
exemplo de como usar os métodos citados com o código anterior.

\begin{lstlisting}
	#include <SocketIOClient.h>
	SocketIOClient socket;
	const char* ssid = NOME DA REDE;
	const char* password = SENHA;
	String host = IP DO SERVIDOR WEB;
	int port = PORTA QUE FOI FORNECIDA AO SERVIDOR WEB;	

	void led(String state) {
	Serial.println("[led] " + state);
	if (state == "\"state\":true") {
	socket.emit("post-informacao","{\"data\":\"1\"}");
	}
	else {
	socket.emit("post-informacao","{\"data\":\"0\"}");
	}
	}

	void setup() {
		WiFi.begin(ssid, password);
  		socket.on("ligar", ligar);  
  		socket.connect(host, port);
	}

	void loop() {
  		socket.monitor();    
	}
\end{lstlisting}


\addtocounter{footnote}{-2}
\footnotetext{Função responsável por emitir os dados para o servidor \textit{web}}
\addtocounter{footnote}{1}
\footnotetext{Função responsável por receber os dados do servidor \textit{web}}
\addtocounter{footnote}{1}
\footnotetext{Função responsável por estabelecer conecção com servidor \textit{web}}


\begin{figure}[h!]
	\includegraphics[width=0.8\textwidth, keepaspectratio=true]{circuito}
	\centering
	\caption[Circuito demonstrativo para comunicação com o \textit{power monitor}]{Circuito demonstrativo para comunicação com o \textit{power monitor}}
	\label{fig:circuito} 
\end{figure}
\FloatBarrier

\section[\textit{Resultados}]{\textit{Resultados}}\label{resultados-sec}


% Capitulo 4
\include{textuais/capitulo4}

% Capitulo 5
\include{textuais/capitulo5}

% Conclusão
\chapter[Conclusão]{Conclusão}
\label{ch:conclusao-cap}

Do início ao fim deste trabalho de conclusão de curso, alguns conceitos foram apresentados
e demonstrados a respeito do consumo de energia elétrica, visando criar um cenário
comparativo a respeito da situação energética do país. Ao longo dos anos o povo brasileiro
ouviu sobre propostas na melhoria do meio ambiente, sustentáveis e até
melhoria na infraestrutura urbana. Porém na maiora das vezes nunca passou de apenas propostas.

Por meio do estudo realizado neste trabalho constatou-se que a leitura do
consumo energético feita em tempo real e disponibilizada para o usuário de uma maneira entendível e direta
desperta a curiosidade e consequentemente a conscientização do mesmo. Apenas mostrar os resultados coletados não seria suficiente
para a conscientização do cidadão brasileiro, a forma com que os resultados são relatados para o consumidor tem um grande impacto, pois
no momento em que a unidade de medida do consumo de energia elétrica é "trocada" do quilowatt-hora para a moeda em circulação no país, o real, 
faz com que todos os brasileiros independente da classe social ou do grau de escolaridade entenda o quanto tem se consumido e disperdiçado em sua residência.

Este trabalho traz uma solução eficiente no monitoramento de energia elétrica, tanto no ambito do \textit{software} como na comunicação com 
\textit{hardware}. O Sistema foi pensado de tal maneira que mesmo em que não exista a presença de um \textit{hardware} para controlar os dispositivos físicos
o utente possa, ainda sim, monitorar os seus gatos por meio das simulações que levam em conta o consumo esperado do dispositivo. A facilidade de estabelecer 
a comunicação entre servidor \textit{web} e outros dispositivos físicos, traz a possibilidade do usuário poder sempre expandir e melhorar as formas
de monitoramento de energia em sua residência. Além disso o trabalho implementa um novo padrão para se mensurar o consumo da energia elétrica nas
residências ,  




% ----------------------------------------------------------
% ELEMENTOS PÓS-TEXTUAIS

% 
\postextual

% Referências bibliográficas
%\addcontentsline{toc}{chapter}{Referências Bibliográficas}

\bibliographystyle{abntex2-alf}
%\bibliographystyle{unsrt}
\bibliography{bibliografia/referencias}

\end{document}
