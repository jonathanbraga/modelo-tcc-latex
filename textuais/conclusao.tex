\chapter[Conclusão]{Conclusão}
\label{ch:conclusao-cap}

Do início ao fim deste trabalho de conclusão de curso, alguns conceitos foram apresentados
e demonstrados a respeito do consumo de energia elétrica, visando criar um cenário
comparativo a respeito da situação energética do país.

Por meio do estudo realizado neste trabalho constatou-se que a leitura do
consumo energético feita em tempo real e disponibilizada para o usuário  
desperta a curiosidade e consequentemente a conscientização do mesmo. Apenas mostrar os resultados coletados não seria suficiente
para a conscientização, a forma com que os resultados são relatados para o consumidor tem um grande impacto, pois
no momento em que a unidade de medida do consumo de energia elétrica é "trocada" do quilowatt-hora para reais (R\$), 
faz com que todos entendam o quanto tem se consumido e desperdiçado em sua residência.

Este trabalho traz uma solução eficiente no monitoramento de energia elétrica, tanto no âmbito do \textit{software} como na comunicação com 
\textit{hardware}. O sistema foi pensado de tal maneira que mesmo em que não exista a presença de um \textit{hardware} para controlar os dispositivos físicos
o usuário possa monitorar os seus gastos por meio das simulações que levam em conta o consumo esperado do dispositivo. A facilidade de estabelecer 
a comunicação entre servidor \textit{web} e outros dispositivos físicos, traz a possibilidade do usuário poder expandir e melhorar as formas
de monitoramento de energia em sua residência.

A troca do quilowatt-hora pela moeda real, faz com que o consumidor entenda a verdadeira situação da sua conta de energia, possibilitando um melhor planejamento
financeiro e o incentivo ao uso consciente. Apesar do sistema atingir os resultados esperados, existem algumas melhorias 
que podem ser consideradas para trabalhos futuros. A primeira seria criar um \textit{hardware} mais robusto, onde adicionaria um Arduino Uno
que contribuiria com mais entradas analógicas e digitais. À adição do Arduino Uno influenciaria na quantidade de eletrodomésticos adicionados 
ao \textit{power monitor}. A segunda melhoria seria em relação ao monitoramento da tensão em um dispositivo ligado, a inclusão de um sensor de tensão 
seria a melhor opção para poder se obter o real consumo de energia elétrica. A última mudança a ser destacada é na parte do 
\textit{software}, melhorias em relação ao servidor \textit{web} como por exemplo: maior segurança nos dados armazenados, cadastro
de mais usuário e segurança nas transações de informação com o \textit{hardware}. Na interface \textit{web} seria necessário um aumento nos \textit{end-points}
que acarretaria em uma maior disponibilidade de dados.

Em conclusão o sistema proposto por este trabalho, supre as necessidades de gerenciamento dos equipamentos eletrônicos
presente em uma residência e de um melhor entendimento em relação aos gastos e desperdícios de energia elétrica.
