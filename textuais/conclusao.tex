\chapter[Conclusão]{Conclusão}
\label{ch:conclusao-cap}

Do início ao fim deste trabalho de conclusão de curso, alguns conceitos foram apresentados
e demonstrados a respeito do consumo de energia elétrica, visando criar um cenário
comparativo a respeito da situação energética do país. Ao longo dos anos o povo brasileiro
ouviu sobre propostas na melhoria do meio ambiente, propostas sustentáveis e até
melhoria na infraestrutura urbana. Porém na maioria das vezes nunca passou de apenas propostas.

Por meio do estudo realizado neste trabalho constatou-se que a leitura do
consumo energético feita em tempo real e disponibilizada para o usuário de uma maneira entendível e direta, 
desperta a curiosidade e consequentemente a conscientização do mesmo. Apenas mostrar os resultados coletados não seria suficiente
para a conscientização do cidadão brasileiro, a forma com que os resultados são relatados para o consumidor tem um grande impacto, pois
no momento em que a unidade de medida do consumo de energia elétrica é "trocada" do quilowatt-hora para a moeda em circulação no país, o real, 
faz com que todos os brasileiros independente da classe social ou do grau de escolaridade entenda o quanto tem se consumido e desperdiçado em sua residência.

Este trabalho traz uma solução eficiente no monitoramento de energia elétrica, tanto no âmbito do \textit{software} como na comunicação com 
\textit{hardware}. O Sistema foi pensado de tal maneira que mesmo em que não exista a presença de um \textit{hardware} para controlar os dispositivos físicos
o utente possa, ainda sim, monitorar os seus gatos por meio das simulações que levam em conta o consumo esperado do dispositivo. A facilidade de estabelecer 
a comunicação entre servidor \textit{web} e outros dispositivos físicos, traz a possibilidade do usuário poder sempre expandir e melhorar as formas
de monitoramento de energia em sua residência. Além disso o trabalho implementa um novo padrão para se mensurar o consumo da energia elétrica nas
residências, no lugar do complexo quilowatt-hora, medida que atualmente é utilizada para quantificação do consumo energético, o \textit{power monitor}
explora uma abordagem mais simples e mais perto do dia a dia de todo brasileiro, mensura todo o consumo de energia elétrica da residência em 
reais, moeda em circulação no país. 

A troca do quilowatt-hora pela moeda real, faz com que o consumidor entenda a verdadeira situação da sua conta de energia, possibilitando um melhor planejamento
financeiros e o incentivo a consumir apenas o que é necessário. Apesar do sistema atingir os resultados esperados, existem algumas melhorias 
que podem ser consideradas para trabalhos futuros. A primeira seria criar um \textit{hardware} mas robusto, onde adicionaria um Arduino Uno
que contribuiria com mais entradas analógicas e digitais, isso influenciaria em uma maior acoplação de relés que por sua vez
influencia na quantidade de eletrodomésticos adicionados ao \textit{power monitor}, o arduino por sua vez se comunicaria via conexão serial 
com o NodeMCU. A Segunda melhoria seria em relação ao monitoramento da tensão em um dispositivo ligado, a inclusão de um sensor de tensão 
seria a melhor opção para poder se obter a potencial real que o dispositivo consome. A última mudança a ser destacada é na parte do 
\textit{software}, melhorias em relação ao servidor \textit{web} como por exemplo, maior segurança nos dados armazenados, possibilidade para criação
de mais usuários, segurança nas transações de informação com o \textit{hardware} e com a interface \textit{web} e um aumento nos \textit{end-points}
que acarretaria em uma maior disponibilidade de dados. Na parte da interface \textit{web} as melhorias seriam direcionadas a correção de \textit{bugs} 
existentes na primeira versão e uma melhor estruturação para que se possibilite uma customização do usuário em relação aos dados em que ele deseja visualizar.

Em conclusão o sistema proposto por este trabalho, supre as necessidades de gerenciamento dos equipamentos eletrônicos
presente em uma residência e de um melhor entendimento em relação aos gastos e desperdícios de energia elétrica.
