\chapter[Embasamento Teórico]{Embasamento Teórico}
\label{ch:cap2}
O \textit{Power Monitor} surgiu da necessidade da conscientização do gasto energético e da melhor compreensão da conta de luz. Baseado nesse conceito,
foram desenvolvido um \textit{software} que permitirá uma fácil comunicação com qualquer equipamento construido que tenha a finalizade de monitorar a energia elétrica.
O sistema traz uma forma mais fácil e próxima do consumidor final de se quantificar a energia elétrica consumido em um estabelecimento. No lugar do Quilowatt-hora, medida que é usada atualmente,
o \textit{software} propõe mensurar o gasto energético em reais (R\$), trazendo a realidade do consumo mensal para mais próximo de cada brasileiro.

Esse capítulo trará os conceitos essenciais para o entendimento do trabalho, descrevendo todas as tecnologias utilizadas no desenvolvimento 
do \textit{software} como do \textit{hardware}.

\section[\textit{Ferramentas e Linguagem}]{\textit{Ferramentas e Linguagem}}\label{ferramenta-linguagem}
No decorrer do desenvolvimento do \textit{software} fez-se uso de algumas tecnologias e linguagens de programação que serão descrita no decorrer
dessa seção.
\subsection[\textit{Node.js}]{\textit{Node.js}}\label{node}
Node.js é um interpretador do código JavaScript (\autoref{js}), com o foco do uso da linguagem do lado do cliente para servidores. Com um objetivo simples
que é ajudar desenvolvedores na criação de aplicações de alta escalabilidade, com códigos capazes de administrar e manipulazar várias conexões simultaneamente
em um único servidor. O \textit{Node.js} é baseado na \textit{runtime} V8 \textit{JavaScript Engine}. Foi desenvolvido por Ryan Danhl em 2009, e o seu desenvolvimento
é mantido pela fundação \textit{Node.js} e \textit{Linux Foundation}. 
\subsection[\textit{JavaScript}]{\textit{JavaScript}}\label{js}
\subsection[\textit{WebSocket}]{\textit{WebSocket}}\label{websocket}
\subsection[\textit{SQL}]{\textit{SQL}}\label{sql}

\section[\textit{Componentes Físicos}]{\textit{Componentes Físicos}}\label{comp-fisico}
\subsection[\textit{ESP8266}]{\textit{ESP8266}}\label{esp}

%NodeJs
%javascript
%jquery
%websocket
%Sql
%Mysql
%esp8266
%componentes do circuito
