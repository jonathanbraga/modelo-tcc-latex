\chapter[Embasamento Teórico]{Embasamento Teórico}
\label{ch:cap2}
\section[\textit{Contextualizacao}]{\textit{Contextualização}}\label{context}
O Consumo de energia elétrica é um dos principais indicadores de desenvolvimento e de qualidade de vida 
de um país. Essw índice é tão importate que reflete diretamente no rítimo de vida de uma populção, pois mostra
se as atividades industriais de uma nação está ou não em um bom rítimo e pode detectar se o comércio está alta,
devido aos bens e serviços que o povo adiquiriu. Analisando os dados \cite{epe-balanco-final}, o consumo de energia
elétrica no Brasil vem crescendo ao longo dos anos, o brasileiro vem consumindo mais energia elétrica, nos últimos
35 anos teve um crescimento médio de 6.72\% dessa demanda 

Itens em latex:
\begin{itemize}
	\item texto 1;
	\item tewd;
	\item texto 3;
\end{itemize}

\lipsum[2-3]

\subsection{Subseção}

\lipsum[2-4]