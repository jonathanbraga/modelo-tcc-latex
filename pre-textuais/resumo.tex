% resumo na língua vernácula (obrigatório)
\setlength{\absparsep}{18pt} % ajusta o espaçamento dos parágrafos do resumo
\begin{resumo}
Braga, J. S. \textbf{Proposta de uma ferramenta de baixo custo para o monitoramento de energia residencial.} 2018. 35 p. Trabalho de Conclusão de Curso
(Engenharia Mecatrônica) - Universidade Federal do Rio Grande do Norte, Natal, 2018.
\vspace*{1cm}


Este trabalho possui como objetivo o desenvolvimento de um sistema domiciliar de monitoramento de energia elétrica de baixo custo, em tempo real.
O sistema tem como base a placa de desenvolvimento NodeMCU e um sensor de corrente que afere o consumo em tempo real de um dispositivo. Através
dos dados coletados e dos cálculos realizados pelo sistema é possível acompanhar o consumo de energia elétrica de uma forma mais fácil.
O intuito é que o constante monitoramento possa trazer uma conscientização da economia de energia.
 
 \noindent
 \textbf{Palavras-chaves}: Economia de energia. Monitoramento. Eficiência Energética. IoT. 
\end{resumo}
% ---
% resumo em inglês
\begin{resumo}[Abstract]
	\begin{otherlanguage*}{english}	
Braga, J. S. \textbf{	Proposal of a low cost tool for residential energy monitoring.} 2018. 35 p. Conclusion work project (Graduate in Mechatronics Engineering) - Federal
University of Rio Grande do Norte, Natal, 2018.
\vspace*{1cm}	
	
The objective of this work is the development of a low-cost real-time home-based energy metering system. The system is based on the NodeMCU development board and a current sensor that measures the real-time consumption of a device. Through the data collected and the calculations made by the system it is possible to monitor the consumption of electricity in an easier way. The intention is that constant monitoring can bring an awareness of the energy economy.
	
	\vspace{\onelineskip}
	\noindent 
	\textbf{Keywords}: Energy saving. Monitoring. Energy Efficiency. IoT
	\end{otherlanguage*}
\end{resumo}